\section{Problem Formulation}
\label{sec:prob}

In this section, we present a pair of dual infinite dimensional linear programs that compute the uncertain backwards reachable set. 
Critically, note that despite the uncertainty being drawn from a distribution at the arrival into each mode, it remains constant throughout that mode. 
As a result, this unknown parameter can be appended to the dynamics of every mode $j$ and treated as a portion of the state-space:
\begin{align}
f_j=\begin{bmatrix}
  \tilde f_j'&\mathbf{0}'_{n_{\theta_j}}
\end{bmatrix}'.
\end{align}
%In this augmented-state-space---henceforth referred to as the state-space of the system---the object of interest still remains the same, $X_{(0,j)},\,\forall j\in \mathcal J$. Furthermore, as the system transitions out of a mode, say at time $\tau_k$ the solution reaches $e_{ij}$, the uncertainty in mode $j$ is not related to the \emph{actual} value of the uncertainty in mode $i$ at $\tau_k$; in fact, the dimensions of the uncertain parameters $n_{\theta_j}$ need not equal $n_{\theta_i}$, much less their distributions.

To address this problem, we rely on the notion of occupation measures, first introduced in \cite{Pitman1977}, to transform the hybrid nonlinear dynamics of the system into a set linear dynamics over measures that can more readily be solved.
Occupation measures can be interested as measuring the time a solution spends in a portion of the state-space. 
For instance, suppose the system enters mode $j$ at $\tau_k$ with the states being initialized as $x(\tau_k)=x_0$ and $\theta(\tau_k)=\theta$.
The occupation measure, \mbox{$\mu_j(\cdot\mid \tau_k,x_0,\,\theta)\in \mathcal M(\mathcal T\times D_j\times \Theta_j)$}, is defined as:
\small
\begin{align}
\hspace*{-1mm}\mu_j(A\times B\times C| \tau_k ,x_0,\theta)=\hspace*{-1.25mm}\int\limits_{\cal T} \hspace*{-1.25mm} I_{A\times B\times C}(t,x(t|\tau_k,x_0,\theta),\theta)dt.
\end{align}
\normalsize
Note the follow relation between the Lebesgue measure on $\mathcal T$ and $\mu_j(\cdot\mid \tau_k,x_0,\theta)$ holds for all $v \in C(\mathcal T\times D_j\times \Theta_j)$:
\begin{align}
\ip{\mu_j(\cdot\mid \tau_k,x_0,\theta),v}=\ip{\lambda_t,v(t,x(t\mid \tau_k,x_0,\theta),\theta)},
\label{eq:mu_lambda}
\end{align}
The occupational measure as defined is a conditional measure -- conditioned on the arrival-time and initial values of the states in that mode.
To consider a set of possible arrival-times and initial conditions, we define the \emph{average occupation measure} by integrating the conditional occupation measure against a measure on the set of possible initial conditions of the mode, $\mu_{s_j} \in  M(\mathcal T\times D_j\times \Theta_j)$:
\begin{align}
\mu_j(A\times B\times C)= \hspace*{-5mm}\int\limits_{\mathcal T\times D_j\times \Theta}\hspace*{-5mm}\mu_j(A\times B\times C\mid \tau_k,x_0,\theta)\,d \mu_{s_j}.
\label{eq:mu_avg}
\end{align}
Observe that by definition, the uncertain variables are independent of the states' initial conditions; hence $\mu_{s_i}\in \mathcal M(\mathcal T\times D_j\times \Theta)$ is expressible as a product measure:
\begin{align}
\mu_{s_j}=\bar\mu_{0_j}\otimes \mu_{\theta_j},\
\end{align}
where $\bar \mu_{0_j}\in \mathcal M(\mathcal T\times D_j)$ is a measure describing the set of initial conditions, and $\mu_{\theta_j}\in \mathcal M(\Theta_j)$ is as in the definition of $\mathcal H$.

Similarly, measures on terminals sets, $\mu_{T_j}\in \mathcal M(X_{(T,j)}\times \Theta_j)$:
\begin{align}
\mu_{T_j}(A\times B)= \hspace*{-4mm}\int\limits_{\mathcal T\times X_{(T,j)}\times \Theta}\hspace*{-4mm}I_{A\times B}(x(T\mid \tau_k,x_0,\theta),\theta)\,d\mu_{s_{j}},
\end{align}
and guards, $\mu_{ G_{e}}\in \mathcal M(\mathcal T\times  G_{(j,k)})$:
\begin{align}
\mu_{G_{(j,k)}}(A\times B\times C)=\hspace*{-5mm}\int\limits_{\mathcal T\times G_{(j,k)}}\hspace*{-5mm}I_{A\times B\times C}(t,x(t\mid \tau_k,x_0,\theta),\theta)\,d\mu_{s_{j}}
\end{align}
for all $(j,k) \in {\cal E}$ can be defined. 
The measures $\mu_{ G_{(j,k)}}$ are supported on the guards of mode $j$ and should be interpreted as the hitting times of the guard.
The {\em final measure} in each mode $j$ can be defined as:
\begin{align}
  \mu_{f_j}=\delta_T\otimes \mu_{T_j}+\sum_{k\in\{l\mid (j,l)\in \mathcal E\}}\mu_{ G_{(j,k)}}.
\label{eq:mu_T}
\end{align}

% Given a set of initial conditions $X_{0}$, the dynamics of the system---under appropriate assumptions---defines a bundle of trajectories of the system states. It is of interest to ensure that this bundle terminates in the desired set $X_T$, making $X_0$ a subset of the BRS; stated differently, it is necessary to relate $\{\mu_{s_j}\}_{j\in \mathcal J}$ with $\{\mu_{f_j}\}_{j\in \mathcal J}$ and the dynamics of the system. 
To compute $X_0$, we relate $\{\mu_{s_j}\}_{j\in \mathcal J}$ with $\{\mu_{f_j}\}_{j\in \mathcal J}$ using the dynamics of the system.
As a first step, define linear operators {$\mathcal L_{ f_j}\colon \mathcal C^1(\mathcal T\times D_j\times \Theta_j)\rightarrow \mathcal C(\mathcal T\times D_j\times \Theta_j)$} as:
\begin{align}
      \mathcal L_{f_j}v=\frac{\partial v}{\partial t}+\ip{\nabla_x v,\tilde f_j}
    \label{eq:Lv}
\end{align}
where $v\in \mathcal C^1(\mathcal T\times D_j\times \Theta_j;\R)$ is an arbitrary test function.
Suppose the system transitioned to mode $j$ at \mbox{$t=\tau_{k-1}$} with the state taking value upon reset $x(\tau_{k-1})$ and $\theta$.
The value of $v$, evaluated along the flow of the system and at $t=\tau_{k}$ is computed using the Fundamental Theorem of Calculus:
\begin{align}
\begin{aligned}
    v& \big(\tau_k,x(\tau_{k}\mid x(\tau_{k-1}),\theta_{k-1})\big)=v(\tau_{k-1},x(\tau_{k-1}),\theta_{k-1})\\
    &+\int_{\tau_{k-1}}^{\tau_{k}}\hspace*{-2mm}\mathcal L_{f}v(t,x(t\mid \tau_{k-1},x(\tau_{k-1}),\theta_{k-1}))\,dt.
\end{aligned}
\label{eq:FTC}
\end{align}
Using Eqn.~(\ref{eq:mu_lambda}), Eqn.~(\ref{eq:FTC}) can be re-written as:
\small
\begin{align}
\begin{aligned}
    v\big(\tau_k,&x(\tau_{k}\mid \tau_{k-1},x(\tau_{k-1}),\theta_{k-1})\big)=v(\tau_{k-1},x(\tau_{k-1}),\theta_{k-1})\\
    &+\ip{\mu_j(\cdot\mid \tau_{k-1},x(\tau_{k-1},\theta_{k-1}),\mathcal L_{ f}v},
\end{aligned}
\end{align}
\normalsize
which can be simplified further by using Eqns.~(\ref{eq:mu_avg})--(\ref{eq:mu_T}):
\begin{align}
  \ip{\mu_{f_j},v}=\ip{\mu_{s_j},v}+\ip{\mu_{j},\mathcal L_{f}v}.
  \label{eq:liouville_1}
\end{align}

Alternatively, using the standard definition of adjoint operators\footnote{A linear operator $\mathcal L$ and its adjoint, $\mathcal L'$, satisfy the following relation:
\begin{align*}
    \ip{\mathcal L'\mu,v}=\ip{\mu,\mathcal Lv}.
\end{align*}
% \begin{align*}
%     \ip{\mathcal L'\mu,v}=\ip{\mu,\mathcal Lv}=\int\limits_{\mathcal X}\mathcal Lv\,d\mu.
% \end{align*}
}, Eqn.~(\ref{eq:liouville_1}) is re-written as:
\begin{align}
\ip{\mu_{f_j},v}=\ip{\mu_{s_j},v}+\ip{\mathcal L'_{f}\mu_{j},v}.
  \label{eq:liouville_2}
\end{align}
\emph{Eqn~(\ref{eq:liouville_2}) defines a linear relation that initial and final measures evolving according to the hybrid dynamics must satsify.}

\par
In the execution of system $\mathcal H$, each mode can be entered in two ways -- at $t=0$; and because of a reset map, at any time $t\in \mathcal T\backslash\{0,T\}$; hence the initial measure in the $(t,x)$-projection can be decomposed as
\begin{align}
  \bar\mu_{0_j}=\delta_0\otimes\mu_{0_j}+\pi_{t,x}\sigma_{0_j}
\end{align}
with $\mu_{0_j}\in \mathcal M(D_j)$ is the measure supported on the initial conditions to the system at $t=0$, and $\sigma_{0_j}\in \mathcal M(\mathcal T\times D_j\times \Theta_j)$ is the measure on initial conditions after the first reset. State resets occur when the states reach the guard and, unless the solution terminates at the guard, for every solution terminating in the support of $\mu_{ G_{(i,j)}},\,\forall (i,j)\in \mathcal E$, there must exist a trajectory originating in the support of $\sigma_{0_j},\,\forall j\in \mathcal J$; that is, $\mu_{ G_{(i,j)}},\forall (i,j)\in \mathcal E$, and $\sigma_{0_j},\forall j\in \mathcal J$, are related.
\par
To see this relation, $\sigma_{0_j}$ is first decomposed into measures corresponding to the source of each arrival state; i.e.
\begin{align}
  \sigma_{0_j}=\sum_{i\in \{k\mid (k,j)\in \mathcal E\}} \sigma_{(i,j)}\otimes \mu_{\theta_j},
\end{align}
where $\sigma_{(i,j)}$ is the measure on initial conditions post reset for all trajectories arriving at mode $j$ from guard $ G_{(i,j)}$ of mode $i$. Upon reaching the guard, the system transitions according to the reset map; in essence, viewing $R_{(j,k)}$ as a nonlinear transformation of the state-space, the relation in Eqn.~(\ref{eq:reset_measure}) between $\sigma_{(i,j)}$ and $\mu_{ G_{(i,j)}}$ is established.
\begin{align}
    \ip{\sigma_{(i,j)},w}=\ip{\pi_{t,x}\mu_{ G_{(i,j)}},w\circ R_{(i,j)}}
    \label{eq:reset_measure}
\end{align}
where $w\in \mathcal C(\mathcal T\times \mathcal X_j)$ and
$$
  \ip{\pi_{t,x}\mu_{ G_{(i,j)}},s}=\ip{\mu_{ G_{(i,j)}},s},\,\forall s\in \mathcal C(\mathcal T\times D_i);
$$
essentially, $\sigma_{(i,j)}$ is a push-forward measure of $\mu_{ G_{(i,j)}}$.
%\begin{remark}
%\label{remark:primal}
%  \red{A note on domain in time and the Liouville's equations}
%\end{remark}
  \subsection{The primal}
  \label{ssec:primal}
  With the constraints expressed in terms of measures, the problem of approximating the BRS is formulated as an infinite-dimensional Linear Program that supremizes the \emph{volume} of the set of initial condition.
  \begin{flalign}\nonumber
  &\sup_{\Lambda}\sum_{j=1}^{n_m}\ip{\mu_{0_j},1}&&&(P)\\\nonumber
  &\text{st.}\\
  &\mu_{s_j}+\mathcal L_{f}'\mu_j=\,\mu_{f_j}&&&\forall j\in \N_{n_m}\label{eq:primal:liouville}\\
  &\mu_{0_j}+\hat\mu_{0,j}=\,\lambda_j&&&\forall j\in \N_{n_m}\\
  &\sum_{j=1}^{n_m}\ip{\mu_{T_j},1}=\,\sum_{j=1}^{n_m}\ip{\mu_{0_j},1}\label{eq:mass_conservation}
  \end{flalign}
  where $\lambda_j$ is the Lebesgue measure supported on $D_j$.
  $$\Lambda=\{\mu_j,\mu_{0_j},\mu_{T_j},\hat\mu_{0_j},\mu_{ G_{(j,k)}}\ge 0,\,\forall j\in \mathcal \N_{n_m},(j,k)\in \mathcal E\}.$$
Variables $\hat\mu_{0_j}\in \mathcal M(D_j)$ are slack variables introduced to enforce a stronger constraint than absolute continuity of $\mu_{0_j}$ wrt. to $\lambda_j$
  \begin{flalign}
  &&&\mu_{0_j}(A)\le \lambda_j(A)&\forall A\subset D_j\label{eq:primal:domination}
    \end{flalign}
  The constraint in Eqn.~(\ref{eq:mass_conservation}) ensures that all trajectories that emanate $\cup_{j\in \mathcal J}\,spt(\mu_{0_j})$ reach $X_{T}$ at $t=T$, and is not {\em stuck} at any of the guards.
  %%%%%%%%%%
  \begin{lemma}
   If $\mu_{0_j},\forall j\in \mathcal J$ is part of the optimal solution of ($P$) then $\bigcup_{j\in \mathcal J}\,spt(\mu_{0_j})$ is the BRS of the system. In addition, the optimal value of ($P$) is equal to $\sum_{j\in \mathcal J}\lambda_j(X_{0_j})$, the sum of \emph{volumes} of the BRSs in each mode.
  \end{lemma}

  \begin{proof}
  Suppose $\sum_{j\in \mathcal J}\lambda_j(spt(\mu_{0_j})\backslash X_{(0,j)})>0$, then by Lemma~\ref{lemma:existence}, there exist trajectories that begin in $\cup_{j\in \mathcal J}(spt(\mu_{0_j})\backslash X_{(0,j)})$ that reach $X_T$; this is a contradiction. Thus,
  \begin{align}
  &\bigcup_{j\in \mathcal J} spt(\mu_{0_j})\subset \bigcup_{j\in \mathcal J} X_{(0,j)},\\
  &\sum_{j\in \mathcal J}\lambda_j(spt(\mu_{0_j}))\le\sum_{j\in \mathcal J}\lambda_j( X_{(0,j)}).
  \label{eq:support_lemma:1}
  \end{align}
    By definition of the BRS, all state trajectories that emanate from a subset of $X_0$ end in $X_T$. That is, for each $j\in \mathcal J$ and initial measure $\mu_{0_j}$, if $spt(\mu_{0_j})\subset X_{(0,j)}$, there exist measures $\mu_{j}$ and $\mu_{f_j}$ that satisfy Eqn.~(\ref{eq:primal:liouville}). Thus the following inequality is true.
    \begin{align}
    \sum_{j\in \mathcal J}\lambda_j(spt(\mu_{0_j}))\ge \sum_{j\in \mathcal J}\lambda_j(X_{(0,j)})
      \label{eq:support_lemma:2}
    \end{align}
   From Eqns.~(\ref{eq:support_lemma:1})\&(\ref{eq:support_lemma:2}), $\cup_{j\in \mathcal J}\,spt(\mu_{0_j})$ is the BRS of the system.
    \par
    That the optimal value of $(P)$ is volume of the BRS follows from Eqn.~(\ref{eq:primal:domination}) and the observation that \mbox{$\lambda_{j}|_{X_{(0,j)}},\forall j\in \mathcal J$} is feasible in ($P$).
  \end{proof}

  \subsection{The dual}
  \label{ssec:dual}
    The dual corresponding to $(P)$ is derived using standard techniques and is presented below.
    \par
    \footnotesize
    \begin{flalign}\nonumber
    &&&\inf \sum_{j\in \N_{n_m}}\ip{\lambda_j,w_j}&(D)\\\nonumber
    &&&\text{st.}\\
    &&&w_j\ge \,0&\forall (x,j)\in \mathcal D\\
    &&& v_j(T,\cdot)+q\ge\, 0 ,\> &\forall (x,j,\theta)\in \mathcal X_T\times \Theta \label{eq:dual:terminal}\\
    &&& - \mathcal L_{\tilde f}v_j\ge\,0 ,\> &\forall (t,x,j,\theta)\in \mathcal T\times\mathcal D\times \Theta\label{eq:dual:lfv}\\
    &&& w_j-\ip{\mu_{\theta_j},v(0,\cdot)}-q\ge \,1 ,\> &\forall (x,j)\in \mathcal D\label{eq:dual:levelset}\\
    &&&v_j\ge\, \ip{\mu_{\theta_k},v_k}\circ R_{(j,k)},&\forall (t,x,\theta,(j,k)),\in \mathcal T\times \mathcal G\times \mathcal E\label{eq:dual:mode_transition}
    \end{flalign}
    \normalsize
    where $q\in \R$, $v_j\in C^1(\mathcal T\times \mathrm M_j\times \Theta_j)$ and $w_j\in C(\mathrm M_j)$.
    \begin{lemma}
      If $(w,v,q)$ is the solution to (D), then the super-level set
      \begin{align}
      \bigcup_{j\in \mathcal J}\,\{x\mid w_j(x)\ge 1\}
      \end{align}
      is an outer approximation of the BRS of the system whose dynamics is described by Alg.~\ref{alg:execution}.
    \end{lemma}
    \begin{proof}
    The approach we adopt to prove this lemma is to construct the projection of the BRS on any mode and show that it is a 1-level set of the appropriate function. To assist in constructing the arguments, assume wlog., that the state trajectory terminates in $X_{(T,j_k)}$ for some $j_k$. The state trajectory must have arrived in mode $j_k$ through a finite sequence of mode-transitions (according to Assumption~\ref{assump:zeno}); wlog., let this sequences of mode-transitions be of length $k$. Suppose the states entered mode $j_k$ at time $\tau_k$, then, from the fundamental theorem of calculus and the constraints in Eqns.~(\ref{eq:dual:terminal})\&(\ref{eq:dual:lfv}), the following inequalities follow.
    \begin{align}
      -q\le v_{j_k}(T,x(T\mid x(\tau_k^+),\theta),\theta)\le v_j(\tau_k,x(\tau_k^+),\theta)\\
      \Rightarrow -q\le \ip{\mu_{\theta_{j_k}},v_{j_k}(\tau_k,x(\tau_k^+),\theta)}
    \end{align}
    By iterative application of the constraint in Eqn.~(\ref{eq:dual:mode_transition}) and finally Eqn.~(\ref{eq:dual:levelset}), it follows that
    \begin{align}
      -q\le&\, \ip{\mu_{\theta_{j_k}},v_{j_k}(t,x,\theta)}\circ R_{(j_{k-1},j_{k})}(\tau_{k},x(\tau_{k}^-))\\
      \le&\, v_{j_{k-1}}(\tau_{k},x(\tau_{k}^-\mid x(\tau_{k-1}^+),\theta),\theta)\\
      \le &\, \ip{\mu_{\theta_{j_{k-1}}},v_{j_k}(\tau_k,x(\tau_{k-1}^+),\theta)}\\\nonumber
      &\,\vdots\\
      \le &\,v_{j_0}(\tau_1,x(\tau_1^-\mid x_0,\theta),\theta)\\
      \le &\,v_{j_0}(0,x_0,\theta)\\
      \le &\,\ip{\mu_{\theta_{j_0}},v_{j_0}(0,x_0,\theta)}\\
      \le &\, w_{j_0}(x_0)-q-1.
    \end{align}
    The final inequality implies that for every trajectory that ends in $X_{(t,j_k)}$, $x(t)=x_0\in D_{j}$  satisfies the condition $w_{x_0}\ge 1$. Thus the set of initial conditions that begin in mode $j_0$ and reach the terminal set projected into mode $j_k$ is given by the super-level set
    \begin{align}
      I_{(j_0,j_k)}=\{x\mid w_{j_0}\ge 1\}.
    \end{align}
    Note that the definition of $I_{j_0,j_k}$ does not depend on the mode in which the terminal set is reached; thus \mbox{$I_{(j_0,\mathcal J)}=X_{(0,j_0)}$}. Finally, by observing that $j_0$ is an arbitrary element of $\mathcal J$, we deduce the stated result.
    \end{proof}
    \begin{lemma}
      There is not gap between (P) and (D).
    \end{lemma}
    \begin{proof}
      The proof follows from \cite[Theorem 3.10]{Anderson1987}, and is similar to \cite[Theorem 2]{henrion2014convex}; it is not presented for brevity.
    \end{proof}
    \begin{remark}
    There are two key aspects of the presentation in this section that deserve re-iteration: (1) by definition, the uncertainties that influence the dynamics can be visualized as a discrete random process with updates to the instantiation of the uncertainty occurring upon entering a new mode; (2) the estimated BRS is the set of initial conditions from which {\em all} trajectories that emanate reach the terminal set for {\em all} possible discrete sequence of uncertainty. As a direct implication of the second point, the solution of the problem is the intersection of the BRS of every possible sequence of uncertainty.
    \end{remark}
