\section{Existence of solutions}
  \begin{lemma}[Existence of solutions]
Let $(\mu_{s_j},\mu_{f_j},\mu_j), j\in \mathcal J$ satisfy Eqn.~(\ref{eq:primal:liouville}). Then, there exists a family of absolutely continuous trajectories starting from $\mu_{s_j}$ such the occupation and final measures in each mode generated by this family of trajectories is equal to $\mu_j$ and $\mu_{f_j}$.
    \label{lemma:existence}
  \end{lemma}

\begin{proof}[Proof (sketch)]
  That the systems under consideration are uncertain is immaterial for, by definition, the uncertainty is constant in every mode and can be considered as a state in the augmented state vector; the augmented state vector in each mode has dynamics $\tilde f_j$. Hence, in the following presentation any dependence on $\theta$ is suppressed.
  \par
  Adopting the approach used by the authors in proving \cite[Lemma 3]{henrion2014convex}, it can be shown that there exists a solution to the continuity equation

  \small
  \begin{flalign}
    \frac{d}{dt}\int_{\mathrm M_j}w(x)\,d\mu_{j}(\tilde x\mid t)=\int_{\mathrm M_j}\nabla_{x}\tilde f\,d\mu_{j}(x\mid t)&&&\forall w\in \mathcal C(\mathrm M_j)
    \label{eq:app:1}
  \end{flalign}
  \normalsize
    where $\mu_j(\cdot\mid t)$ is the stochastic kernel of $\mu_j$ given $t$.\\
  By using test functions of the variety $v(t,\tilde x)=\varphi(t)\phi(\tilde x)$ in Eqn.~(\ref{eq:primal:liouville}),
  we get the following
  \begin{align}
    \varphi(T)a-\varphi(0)b=\int_{\mathcal T} \varphi(t)c(t)+\varphi(t)(d+e)(t)\,d\lambda_t(t)
        \label{eq:app:2}
  \end{align}
  where 
      $a:=\int_{X_{T,j}}\phi(x)\,d\mu_{f_j}$;  $b:=\int_{X_{0,j}}\phi(x)\,d \mu_{0_j}$; \\
      $c(t):=\int_{{\mathrm M}_j}\phi(\tilde x)\,d\mu_{j}(x\mid t)$; $d:=\int_{{\mathrm M}_j}\nabla_{x}\phi(\tilde x)\tilde f\,d\mu_{j}(x\mid t)$; \\
      and $e(t):=\int_{\Sigma_j}\phi(\tilde x)\,\sigma_{j}(x\mid t)$.\\
  The $d\lambda_t$ a.e. unique solution to Eqn.~(\ref{eq:app:2}) is shown to be
  \begin{align}
    c(t)=b+\int_{\mathcal T}(d+e)(t)\,d\lambda_t,
  \end{align}
  and using the separability of $\mathcal C(\mathrm M_j)$ and the Riesz-representation theorem, that there is a representing measure $\mu_j$
\end{proof}
%
%  \begin{proof}[Proof (sketch)]
%  Given the cyclical definition of systems in $\mathfrak{U}$, it suffices to show that solutions exists in any one mode; wlog., let us consider mode $j$. In addition, wlog. let the time when the trajectory is begins in the support of $\check\mu_{0_j}$ be $\tau$. To assist in the ensuing presentation, define the dynamics of the system in mode $j$ as the following
%  \begin{align}
%    \bar f_j(x,\theta)=\begin{cases}
%      \tilde f_j(x,\theta)& (x,\theta)\not\in \bigcup\limits_{k\in \{l\mid (j,l)\in \mathcal E\}}\mathcal G_{(j,k)}\\
%      0& \text{o.w.}
%    \end{cases}
%  \end{align}
%    If the beginning time in mode $j$ is $\tau$, by Lemma 3 in \cite{henrion2014convex}, given the above dynamics, there exists a family of trajectories that reaches $\mu_{f,j}$. Trajectories that reach a guard before $t=T$, and get stuck would be reset and the system enters another mode wherein similar arguments can be used to show the existence of absolutely continuous trajectories.
%  \end{proof}
