\section{Existence of solutions}
  \begin{lemma}[Existence of solutions]
Let $(\mu_{s_j},\mu_{f_j},\mu_j), j\in \mathcal J$ satisfy Eqn.~(\ref{eq:primal:liouville}). Then, there exists a family of absolutely continuous trajectories starting from $\mu_{s_j}$ such the occupation and final measures in each mode generated by this family of trajectories is equal to $\mu_j$ and $\mu_{f_j}$.
    \label{lemma:existence}
  \end{lemma}
%
%  \begin{proof}[Proof (sketch)]
%  Given the cyclical definition of systems in $\mathfrak{U}$, it suffices to show that solutions exists in any one mode; wlog., let us consider mode $j$. In addition, wlog. let the time when the trajectory is begins in the support of $\check\mu_{0_j}$ be $\tau$. To assist in the ensuing presentation, define the dynamics of the system in mode $j$ as the following
%  \begin{align}
%    \bar f_j(x,\theta)=\begin{cases}
%      \tilde f_j(x,\theta)& (x,\theta)\not\in \bigcup\limits_{k\in \{l\mid (j,l)\in \mathcal E\}}\mathcal G_{(j,k)}\\
%      0& \text{o.w.}
%    \end{cases}
%  \end{align}
%    If the beginning time in mode $j$ is $\tau$, by Lemma 3 in \cite{henrion2014convex}, given the above dynamics, there exists a family of trajectories that reaches $\mu_{f,j}$. Trajectories that reach a guard before $t=T$, and get stuck would be reset and the system enters another mode wherein similar arguments can be used to show the existence of absolutely continuous trajectories.
%  \end{proof}
