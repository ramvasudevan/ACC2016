\section{Introduction}
  
Computing the set of configurations that are able to safely reach a desired configuration is critical to ensuring the correct performance of a system in dynamic environments where deviations from planned behavior are to be expected.
Given its potential, many methods have been proposed to efficiently compute this set of configurations, which is usually referred to as the \emph{backwards reachable set}, for deterministic systems.
Unfortunately, the effect of intermittent contact with the world, especially in fluctuating environments, is difficult to model deterministically. 
A roboticist, for example, may be tasked with ensuring that a control for a legged robot beginning from an initial configuration is able to safely reach a desired goal; however, limitations in sensing or environment variability may render exact modeling of terrain height or friction infeasible.
Few numerical methods exist to tractably compute the backwards reachable set of a dynamical system in spite of parametric model uncertainty while providing systematic guarantees on the computed set. 

This paper leverages a method developed in several recent papers \cite{henrion2014convex,majumdar2014convex,shia2014convex} that describe the evolution of trajectories of a deterministic hybrid dynamical system using measures to describe the evolution of a hybrid dynamical system with parametric uncertainty as a linear equation over measures. 
As a result of this characterization, the set of configurations that are able to reach a target set despite parametric uncertainty, called the \emph{uncertain backwards reachable set}, can be computed as the solution to an infinite dimensional linear program over the space of nonnegative measures. 
To compute an approximate solution to this infinite dimensional linear program, a sequence of finite dimensional relaxations semidefinite programs are constructed that satisfy an important property:
each solution to this sequence of semidefinite programs is an outer approximation to the uncertain backwards reachable set with asymptotically vanishing conservatism.

Several researchers have attempted to systematically construct controllers that can cope with uncertainty in terrain height. 
One approach that has seen success in simulation relies upon discretizing the dynamics according to the step height by considering several possible terrain step sizes \cite{byl2008metastable,saglam2013switching}. 
The effect of walking over these different terrain step sizes is transformed into a Markov Chain with the addition of a discrete state corresponding to falling.
To synthesize a stabilizing controller, the parameters of a simple controller are optimized to find the largest possible mean first passage time to arrive at the falling state. 
This method however is unable to certify the safety of different configurations of a legged robotic system under the generated control. 
Another recent approach considered a single numerical nonlinear optimization program to find a controller that minimized the effect of disturbances across a range of terrain heights \cite{dai2012optimizing}.
To accomplish this objective, the researchers linearized the dynamics around the trajectory dictated by the control input decision variable and penalized the LQR cost-to-go for this linearization at the step time. 
If the optimization program is feasible, its output is a controller that would behave robustly across the range of terrain in a neighborhood of the linearization; however, since the neighborhood where this linearization is valid is not explicitly specified, this method is unable to certify whether a specific configuration of a legged robot can be safely controlled.
More recently, an approach that generates a single control input using a nonlinear program that is constrained to work across a variety of hand selected terrains has been proposed and experimentally validated \cite{griffin2015}. 
For specific configurations and terrain profiles, this method is able to certify the safety of a legged locomotion system since specific trajectories discovered by the optimization under specific terrains are able to walk safely; however, little can be said about arbitrary configurations or step sizes or points in neighborhoods of the discovered configurations. 

More broadly, 

\vspace{1mm}
Uncertainty in the dynamics of these legged robots can also appear due to uncertainty associated with imperfect knowledge of terrain friction or the continuous dynamic model. 
Since controllers from either category are not certified to perform safely in the presence of these more general forms of uncertainty, their translation to real-world systems is still demanding and usually requires stringent model and scenario knowledge followed by thorough testing. 
Controllers designed using exhaustive tuning forfeit the ability to readily generalize lessons learned from controlling one legged robot in a single scenario to other scenarios and other legged robotic systems.
As a result, progress in controlling legged robotic locomotion has been gradual.

The remainder of the paper is organized as follows: 
Section \ref{sec:preliminaries} introduces the notation used in the remainder of the paper, the class of systems under consideration, and the backwards reachable set problem under parametric uncertainty; 
Section \ref{sec:prob} describes how the backwards reachable set under parametric uncertainty is the solution to an infinite dimensional linear program; 
Section \ref{sec:implementation} constructs a sequence of finite dimensional semidefinite programs that outer approximate the infinite dimensional linear program with vanishing conservatism; 
Section \ref{sec:examples} describes the performance of the approach on a pair of examples; 
and, Section \ref{sec:conclusion} concludes the paper. 