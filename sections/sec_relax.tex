\section{Numerical Implementation}
\label{sec:implementation}

In this section, a sequence of Semidefinite Programs (SDP)s that approximate the solution to the infinite dimensional primal and dual defined in Secs.~\ref{ssec:primal} and \ref{ssec:dual} are introduced.
This sequence of relaxations is constructed by characterizing each measure using a sequences of moments\footnote{The $n$th moment of a measure ($\mu$) is obtained by evaluating the following expression
  $$y_{\mu,n}=\ip{\mu,x^n}.$$}
and assuming the following:
\begin{assump}
The vector field in each mode and reset map between modes is a polynomial.
Moreover the domain, the value of uncertainties, the guard, and the target set in each mode is a semi-algebraic set.
  \label{assump:poly}
\end{assump}
Recall that polynomials are dense in the set of continuous functions by the Stone-Weierstrass Theorem so this assumption is made without too much loss of generality.

Under this assumption, given any finite $d$-degree truncation of the moment sequence of all measures in the primal $(P)$, a primal relaxation, $(P_d)$, can be formulated over the moments of measures to construct an SDP.
The dual to $(P_d)$, $(D_d)$, can be expressed as a sums-of-squares (SOS) program by considering $d$-degree polynomials in place of the continuous variables in $D$.

To formalize this dual program, first note that a polynomial $p \in \R[x]$ is SOS or $p \in \text{SOS}$ if it can be written as $p(x) = \sum_{i=1}^m q_i^2(x)$ for a set of polynomials $\{q_i\}_{i=1}^m \subset \R[x]$.
Note efficient tools exist to check whether a finite dimensional polynomial is SOS using SDPs~\cite{parrilo2000structured}.
Next, suppose we are given a semi-algebraic set $A = \{x \in \R^n \mid h_{i}(x) \geq 0, h_i \in \R[x], \forall i \in \N_m \}$.
We define the $d$-degree {\em quadratic module} of $A$ as:
\begin{align}
  \begin{split}
  Q_d(A)=\bigg\{q\in \R_d[x]\,\bigg|\, \exists \{s_k\}_{k \in N_m \cup \{0\}} \subset \text{SOS s.t. } \\ q=s_0+\sum_{k\in \N_{m}}h_{k}s_k \bigg\}
  \end{split}
\end{align}

\noindent The $d$-degree relaxation of the dual, $D_d$, can be written as:
  \begin{flalign}\nonumber
    & & \inf_{\Xi_d} \hspace*{0.1cm} & \sum_{j\in \mathcal J}\int_{D_j}w_j(x)\,d\lambda_j(x) && \hspace*{-0.3cm} (D_d) \nonumber\\
    & & \text{st.} \hspace*{0.1cm} & w_j^d\in Q_d(X_{(T,j)}) && \hspace*{-0.3cm} \forall j\in \mathcal J \nonumber\\
    & & & v_j^d(T,\cdot)+p\in Q_d(D_{j} \times \Theta_j) && \hspace*{-0.3cm} \forall j\in \mathcal J \nonumber\\
    & & & -\mathcal L_{f_j}v_j^d\in Q_d(\mathcal T\times D_{j} \times \Theta_j) && \hspace*{-0.3cm} \forall j\in \mathcal J \nonumber\\
    & & & w_j^d-\ip{\mu_{\theta_j},v_j^d(0,\cdot)}-p-1\in Q_d(D_j) && \hspace*{-0.3cm} \forall j\in \mathcal J \nonumber\\
    & & & v_j^d-\ip{\mu_{\theta_k},v_k^d}\circ R_{(j,k)}\in Q_d(\mathcal T \times D_j \times \Theta_j) && \hspace*{-0.3cm} \forall (j,k)\in {\cal E} \nonumber
  \end{flalign}
where $\Xi_d=\Big\{ \big(\{v^d_j\}_{j\in {\cal J}},\{w^d_j\}_{j \in {\mathcal J}},p \big) \in \bigtimes_{j\in {\cal J}}\R_d[t,x,\theta]$ $\bigtimes_{j \in {\cal J}} \R_d[x]\times\R \Big\}$.
A primal can similarly be constructed, but the solution to the dual can be used directly generate a sequence of outer approximations to the uncertain backwards reachable set:
\begin{lemma}
	For each $d \in \N$ and $j \in {\cal J}$, let $w_{j_d}$ denote the $j$-slice of the $w$-component of the solution to $D_d$. Then ${ X}_{(0,j_d)} = \{x \in D_j \mid w_{j_d}(x) \geq 1 \}$ is an outer approximation to ${ X}_{(0,j)}$ and $\lim_{d\to\infty}\lambda_{n_j}({ X}_{(0,j_d)} \backslash { X}_{(0,j)}) = 0$.
\end{lemma}
\begin{proof}
  The proof to this lemma is an extension of Theorems 5--7 in \cite{shia2014convex} given Lemma~\ref{lemma:dual_outerapprox}.
\end{proof}
