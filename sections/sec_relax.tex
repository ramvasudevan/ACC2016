  \section{Numerical Implementation}
\label{sec:implementation}
  The infinite-dimensional problems described in Secs.~\ref{ssec:primal} and \ref{ssec:dual} are hard to implement and solve directly. In this section, a sequence of {\em relaxed} SDPs---that contains a sub-sequence whose optimal values converges to the optimal value of the problems introduced in Secs.~\ref{ssec:primal} and \ref{ssec:dual}---is introduced.
  \par
  The fundamental idea behind this sequence of relaxations is that measures supported on a compact can be characterized by their moments\footnote{The $n$th moment of a measure ($\mu$) is obtained by evaluating the following expression
  $$y_{\mu,n}=\ip{\mu,x^n}.$$
  By this definition, the mean of a probability distribution (read probability measure) is $y_\mu^1$ and its variance is $y_{\mu}^2-(y_{\mu}^1)^2$.}. Similar to Taylor approximations of functions, longer sequence of moments (higher the order of moments considered) provide a finer approximation of the measure.
  \par
  Since polynomials are dense in set of continuous functions, we restrict our focus on members of $\mathfrak{U}$ that have a polynomial vector-field in each mode and have domains defined by semi-algebraic sets. For such systems, given any finite $d$-degree truncation of the moment sequence of all measures in the primal $P$, a new problem, $P_d$, can be formulated over the moments of measures. This new problem is a Semi-definite Program (SDP). The dual to $P_d$, $D_d$, can be expressed as a sub-of-squares program (SOS program) by considering $d$-degree polynomials in place of the continuous variables in $D$.
  \par
  For each semi-algebraic set $A$, let $(h_{a_i})$ be a collection of polynomials that define the set. For every semi-algebraic set $A$, let its $d$-degree {\em quadratic module} be defined as follows
  \begin{align}
  \begin{split}
  Q_d(A)=\bigg\{q\in \R_d[x]\,\bigg|\, \exists s_k\in \R_{\le d}[x], \text{ SOS },k\in \N_{n_{a}\cup\{0\}},\\
  q=s_0+\sum_{j\in \N_{n_{a}}}h_{a_j}s_j\bigg\}
  \end{split}
  \end{align}
  Using the above notation, the $d$-degree relaxation of the dual, $D_d$, is presented below.
%\par
%  \small
  \begin{flalign}\nonumber
    &\inf_{\Xi_d} \sum_{j\in \mathcal J}\int_{\mathrm M_j}w_j\,d\lambda_j&(D_d)\\\nonumber
    &\text{st.}\\
    & w_j^d\in Q_d(X_{(T,j)})& \forall j\in \mathcal J\\
    & v_j^d(T,\cdot)+p\in Q_d(\mathrm M_{j})&\forall j\in \mathcal J\\
    &-\mathcal L_{f_j}v_j^d\in Q_d(\mathcal T\cup \mathrm M_{j})&\forall j\in \mathcal J\\
    & w_j^d-\ip{\mu_{\theta_j},v_j^d(0,\cdot)}-p-1\in Q_d(\mathrm M_j)&\forall j\in \mathcal J\\
    & v_j^d-\ip{\mu_{\theta_k},v_k^d}\circ R_{(j,k)}\in Q_d(\mathcal T,\mathrm M_j)& \forall (j,k)\in \Upsilon
  \end{flalign}
%  \normalsize
  where $\Xi_d=\{v^d_j,w^d_j,p\}\in (\R_d[t,x,\theta])^{n_m}\times (\R_d[x])^{n_m}\times\R$, \mbox{$\Upsilon=\{(a,b)\mid a\in \mathcal J,\,(a,b)\in \mathcal E\}$} and the other variables are from the given hybrid system $\mathcal H$.
  \begin{lemma}
    The sequence $(\cup_{j\in \mathcal J}\{x\mid w_j^d\ge 1\})_d$ has a convergent sub-sequence of outer approximations of the BRS.
  \end{lemma}
\begin{proof}
  The proof to this lemma is a compilation of proofs to Thms. 5--7 in \cite{shia2014convex}; it is not reproduced for brevity.
\end{proof}
